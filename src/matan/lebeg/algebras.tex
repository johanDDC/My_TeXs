\definecolor{wwwwff}{rgb}{0.4,0.4,1}
\definecolor{qqqqff}{rgb}{0,0,1}
\definecolor{ttttff}{rgb}{0.2,0.2,1}
\definecolor{qqqqzz}{rgb}{0,0,0.6}
\definecolor{qqqqcc}{rgb}{0,0,0.8}

\section{Алгебры.}

\subsection{Определения.}
Пусть множество $X$ не пусто.
\begin{definition}
    \it{Алгеброй подмножеств множества $X$} (алгеброй) называется набор $\mathcal{A} \subset 2^X$, если выполнены следующие свойства:
    \begin{enumerate}
        \item $X, \nothing \in \mathcal{A}$.
        \item $A, B \in \mathcal{A} \implies A \setminus B, A \cap B, A \cup B \in \mathcal{A}$.
    \end{enumerate}
\end{definition}
\begin{definition}
    \it{$\sigma$-алгеброй} называется набор $\mathcal{A} \subset 2^X$, если выполнены следующие свойства:
    \begin{enumerate}
        \item $\mathcal{A}$ является алгеброй.
        \item Если множества $A_1, A_2, \ldots \in \mathcal{A}$, то их счётные объединение и пересечение также
        там лежат: $\bigcap_{j = 0}^{\infty} A_j \in \mathcal{A}, \bigcup_{j = 0}^{\infty} A_j \in \mathcal{A}$
    \end{enumerate}
\end{definition}

\subsection{Примеры $\sigma$-алгебр.}
\begin{example}{(Собственные $\sigma$-алгебры.)\\}
    Наборы $\{\nothing, X\}$ и $2^X$ очевидно являются $\sigma$-алгебрами.
\end{example}

\begin{example}
    Пусть $B \subset X$ --- произвольно подмножество. Тогда набор $\{\nothing, B, X \setminus B, X\}$ является
    $\sigma$-алгеброй. Ну действительно: $\nothing$ и $X$ в нём лежит, объединение любых множеств из набора, как
    конечное, так и счётное снова даст множество из набора (проверяется перебором всех случаев). Аналогично для
    пересечения и разности.
\end{example}

\begin{example}
    \label{alg_example1}
    Пусть $X = \NN$. Рассмотрим следующий набор:
    $\mathcal{A} = \{\text{конечные множества (включая $\nothing$) и дополнения к конечным}\}$. Очевидно, что $\nothing$ и $\NN$
    в нём лежит ($\NN = \overline{\nothing}$). Кроме того при конечном объединении конечных множеств мы получаем
    конечное множество. При конечном объединении дополнений к конечным множествам мы получим дополнение к некоторому
    конечному множеству. Аналогично для конечного объединения конечного множества и его дополнения. Аналогично все
    возможные разности так же снова будут лежать в этом наборе. При этом отдельно для любого множества $A$ из нашего
    набора разность вида $\NN \setminus A$ можно воспринимать как дополнение $A$ ($\overline{A}$), и они все будут
    лежать в наборе. Кроме того наш набор так же замкнут относительно конечного операции пересечения, т.к. мы уже выяснили, что
    всевозможные объединения и дополнения множеств из набора снова лежат в наборе, по закону де Моргана операция
    пересечения выражается через операции объединения и дополнения.\\
    Таким образом набор $\mathcal{A}$ является алгеброй.\\
    Рассмотрим подмножество $2\NN$ --- подмножество чётных чисел. Оно являет счётным объединением конечных множеств
    ($2\NN = \bigcup_{i = 0}^{\infty} 2i$), но не является дополнением ни к какому конечному множеству, а значит не
    лежит в $\mathcal{A}$. Таким образом $\mathcal{A}$ не является $\sigma$-алгеброй.
\end{example}

\begin{example}
    \label{length example}
    Пусть $X = [0, 1]$. Рассмотрим набор $\mathcal{A} = \{\text{конечное объединение попарно непересекающихся
    промежутков на $X$}\}$, где под промежутком понимаются интервалы, отрезки и полуинтервалы.\\
    Очевидно, что сам $X$ там лежит. Пустое множество можно обозначить как $(a, a)$ например, оно там также лежит.
    Очевидно, что конечное объединение конечных объединений в свою очередь будет являться конечным объединением, и
    будет лежать в наборе. Кроме того понятно, что разность двух конечных объединений промежутков может быть либо
    пустым множеством, либо конечным объединением, и, соответственно, в обоих случаях она будет лежать в наборе.
    Рассмотрим теперь операцию пересечения двух конечных объединений промежутков:\\
    Пусть у нас есть два конечных объединения попарно не пересекающихся промежутков: $\bigsqcup_{n = 1}^N I_n$
    и $\bigsqcup_{k = 1}^M J_k$. Тогда имеем:
    \[
        \bigsqcup_{n = 1}^N I_n \cap \bigsqcup_{k = 1}^M J_k =
        \bigsqcup_{n = 1}^N \left( I_n \cap \bigsqcup_{k = 1}^M J_k \right) =
        \bigsqcup_{n = 1}^N \bigsqcup_{k = 1}^M I_n \cap J_k
    \]
    Т.к. никакие промежутки у нас попарно не пересекаются, то $I_n \cap J_k$ может быть равно
    \begin{itemize}
        \item $\nothing$, если $I_n \neq J_k$;
        \item некоторому отрезку, если $I_n = J_k$;
    \end{itemize}
    В любом из двух случаев мы снова получаем множество из $\mathcal{A}$. Таким образом $\mathcal{A}$ является
    алгеброй.\\
    Чтобы показать, что $\mathcal{A}$ не является $\sigma$-алгеброй, покажем, что ...?
\end{example}
Именно с последнего примера мы и начнём вникать в понятие \it{длины множества}.

\subsection{Порождённая $\sigma$-алгебра}
\begin{proposal}
    Пусть наборы $\mathcal{A}, \mathcal{B}$ --- $\sigma$-алгебры. Тогда $\mathcal{A} \cap \mathcal{B}$
    так же является $\sigma$-алгеброй.
\end{proposal}
\begin{proof}
    Пусть $\mathcal{F} = \mathcal{A} \cap \mathcal{B}$. Очевидно, что $\nothing$ и $X \in \mathcal{F}$,
    т.к. они лежали в исходнных $\sigma$-алгебрах. Пусть теперь $A, B \in \mathcal{F}$. Значит $A, B$
    лежали так же в исходных $\sigma$-алгебрах, а значит в них же лежали их пересечение, объединение и
    разность, а значит они так же лежат и в $\mathcal{F}$. Пусть теперь $A_1, A_2, \ldots \in \mathcal{F}$
    --- произвольный счётный набор множеств. Значит этот набор также лежал в $\mathcal{A}$ и $\mathcal{B}$,
    а значит там же лежали их счётные объединение и пересечение (т.к. $\sigma$-алгебры), а значит они
    так же лежат в $\mathcal{F}$.
\end{proof}
\begin{corollary}
    Пересечение произвольного числа $\sigma$-алгебр является $\sigma$-алгеброй.
\end{corollary}
Пусть $S$ --- какой-то набор подмножеств множества $X$.
\begin{definition}
    $\sigma(S)$ называется $\sigma$-алгеброй, \it{порождённой} $S$. $\sigma(S)$ есть пересечение всех
    $\sigma$-алгебр, содержащих $S$.
\end{definition}
\begin{example}
    Пусть $S = \{B\}$. Тогда $\sigma(S) = \{\nothing, B, X \setminus B, X\}$
\end{example}
%    \begin{example}
%        Пусть $S = \{B_1, B_2, \ldots, B_n\}$.
%    \end{example}
\begin{example}
    (Борелевские $\sigma$-алгебры)

    $\mathcal{B}[0, 1]$ --- $\sigma$-алгебра порождена всеми возможными промежутками на $[0 ,1]$.

    $\mathcal{B}(\RR)$ --- $\sigma$-алгебра порождена всеми возможными промежутками на $\RR$.

    Такие $\sigma$-алгебры называются \it{Борелевскими $\sigma$-алгебрами}. Аналогично определяются
    Борелевские $\sigma$-алгебры в многомерном случае.
\end{example}
\begin{proposal}
    \[
        \mathcal{B}(\RR) = \sigma((a, +\infty))
    \]
\end{proposal}
\begin{proof}
    Включение $\mathcal{B}(\RR) \supseteq \sigma((a, +\infty))$ очевидно.\\
    Докажем $\mathcal{B}(\RR) \subseteq \sigma((a, +\infty))$.
    Докажем, что $\sigma((a, +\infty))$ содержит так же все промежутки. Понятно, что для этого достаточно
    показать, что $\sigma((a, +\infty))$ содержит всевозможные лучи во всех концах. Обозначим
    $\sigma((a, +\infty))$ как $\mathcal{A}$. Тогда:
    \begin{itemize}
        \item $\mathcal{A}$ содержит лучи вида $(-\infty, a]$, т.к. они являются дополнением к лучам вида
        $(a, +\infty)$, а все дополнения лежат в $\mathcal{A}$.
        \item $\mathcal{A}$ содержит лучи вида $(-\infty, a)$, как счётное объединение лучей вида
        $\left( -\infty, a - \frac{1}{n} \right]$.
        \item $\mathcal{A}$ содержит лучи вида $[a, \infty)$, как дополнение к лучам вида $(-\infty, a)$.
    \end{itemize}
    Таким образом мы показали, что $\mathcal{A}$ содержит все возможные виды лучей. Далее применяя операцию
    разности множеств можно получить всевозможные промежутки, а это в точности $\mathcal{B}(\RR)$.
\end{proof}
Пусть $\mathcal{A}, \mathcal{B}$ --- две $\sigma$-алгебры.
\begin{definition}
    \it{Произведением} $\sigma$-алгебр $\mathcal{A}$ и $\mathcal{B}$ называется $\sigma$-алгебра
    $\mathcal{C}$, такая что
    \[
        \mathcal{A} \otimes \mathcal{B} = \mathcal{C} =
        \sigma\left( \left\{ A \times B \colon A \in \mathcal{A}, B \in \mathcal{B} \right\} \right)
    \]
\end{definition}
\begin{problem}
    Возьмём Борелевскую $\sigma$-алгебру на плоскости $\mathcal{B}(\RR^2)$. По аналогии с Борелевской
    $\sigma$-алгеброй на прямой можно показать, что
    \[
        \mathcal{B}(\RR^2) = \sigma(\{B_i \subset \RR^2 \colon B_i \text{ --- открытое множество}\})
    \]
    \begin{enumerate}
        \item Покажите, что $\mathcal{B}(\RR^2)$ совпадает с $\sigma$-алгеброй, порождённой всеми
        открытыми прямоугольниками в $\RR^2$, где под ''открытыми прямоугольниками`` понимается
        декартово произведение интервалов.
        \item Докажите, что $\mathcal{B}(\RR^2) = \mathcal{B}(\RR) \otimes \mathcal{B}(\RR)$.
    \end{enumerate}
    \it{Решение:}

    1) Введём обозначение: $\mathcal{O} = \sigma(\{B_i \subset \RR^2 \colon B_i \text{ --- открытое множество}\})$,
    $\sigma(\Box)$ --- $\sigma$-алгебра, порождённая всеми открытыми прямоугольниками. Тогда очевидно
    включение $\mathcal{O} \supset \sigma(\Box)$. Покажем теперь включение в обратную сторону.

    По определению: некоторое множество называется открытым, если каждая точка входит в него с некоторым
    открытым шаром. Пусть $U \in \sigma(\Box)$ --- открытое множество. И пусть $a \in U$ --- произвольная
    точка. Тогда, т.к. $U$ --- открытое множество, то $\exists ~ B(a, r ) \subset U$ --- открытый шар с
    центром в точке $a$ некоторого радиуса $r$. Но в каждый шар можно вписать прямоугольник, в котором
    данная точка будет содержаться. Впишем прямоугольник в открытый шар $B$, и будем уменьшать его до тех
    пор, пока все вершины прямоугольника не станут рациональными точками. Более формально:
    $\forall ~ a \in U ~ \exists ~ \Box$ c рациональными вершинами $ \colon a \in \Box \subset U$. А
    значит, $U$ разбивается на пересечение таких квадратиков $\Box$, причём, т.к. их вершины являются
    рациональными точками, то таких квадратиков не более чем счётно, а значит $\mathcal{O} \subset \sigma(\Box)$,
    и $\mathcal{O} = \sigma(\Box)$.
    \begin{align*}
        &\definecolor{qqffcc}{rgb}{0,1,0.8}
        \definecolor{qqqqzz}{rgb}{0,0,0.6}
        \definecolor{qqqqcc}{rgb}{0,0,0.8}
        \definecolor{ccqqqq}{rgb}{0.8,0,0}
        \definecolor{ffcccc}{rgb}{1,0.8,0.8}
        \begin{tikzpicture}[line cap=round,line join=round,>=triangle 45,x=20,y=20]
            \draw[->,color=black] (-0.63,0) -- (13.11,0);
            \foreach \x in {,1,2,3,4,5,6,7,8,9,10,11,12,13}
            \draw[shift={(\x,0)},color=black] (0pt,-2pt);
            \draw[->,color=black] (0,-0.27) -- (0,10.88);
            \foreach \y in {,1,2,3,4,5,6,7,8,9,10}
            \draw[shift={(0,\y)},color=black] (2pt,0pt) -- (-2pt,0pt);
            \clip(-0.63,-0.27) rectangle (13.11,10.88);
            \fill[color=ffcccc,fill=ffcccc,fill opacity=0.1] (3.86,7.22) -- (5.58,7.78) -- (10.86,5.28) -- (8.6,4.48) -- (8.92,1.78) -- (7.74,0.28) -- (4.9,1.72) -- (1.82,2.56) -- cycle;
            \fill[color=ccqqqq,fill=ccqqqq,fill opacity=0.15] (2.87,3.36) -- (4.77,3.36) -- (4.77,5.25) -- (2.87,5.25) -- cycle;
            \fill[color=qqffcc,fill=qqffcc,fill opacity=0.4] (3,5) -- (4.45,5) -- (4.45,3.9) -- (3,3.9) -- cycle;
            \draw [shift={(3.19,4.74)},color=ffcccc,fill=ffcccc,fill opacity=0.1]  plot[domain=1.31:4.15,variable=\t]({1*2.57*cos(\t r)+0*2.57*sin(\t r)},{0*2.57*cos(\t r)+1*2.57*sin(\t r)});
            \draw [shift={(6.91,3.77)},color=ffcccc,fill=ffcccc,fill opacity=0.1]  plot[domain=0.37:1.89,variable=\t]({1*4.23*cos(\t r)+0*4.23*sin(\t r)},{0*4.23*cos(\t r)+1*4.23*sin(\t r)});
            \draw [color=ffcccc] (3.86,7.22)-- (5.58,7.78);
            \draw [color=ffcccc] (10.86,5.28)-- (8.6,4.48);
            \draw [color=ffcccc] (8.6,4.48)-- (8.92,1.78);
            \draw [color=ffcccc] (8.92,1.78)-- (7.74,0.28);
            \draw [color=ffcccc] (7.74,0.28)-- (4.9,1.72);
            \draw [color=ffcccc] (4.9,1.72)-- (1.82,2.56);
            \draw [dash pattern=on 2pt off 2pt,color=qqqqzz] (3.82,4.31) circle (1.34cm);
            \begin{scriptsize}
                \draw[color=black] (7,4.93) node {\boldmath{$U$}};
                \fill [color=qqqqcc] (3.82,4.31) circle (1.5pt);
                \draw[color=qqqqcc] (3.91,4.47) node {a};
                \draw[color=qqqqzz] (2.72,5.48) node {$B(a, r)$};
            \end{scriptsize}
        \end{tikzpicture}\\
        &\text{Здесь $U$ --- открытое множество, $a$ --- произвольная точка в нём,}\\
        &\text{$B(a, r)$ --- открытый шар точки $a$,}\\
        &\text{красный квадрат --- произвольный вписаный в шар прямоугольник,}\\
        &\text{зелёный квадрат --- прямоугольник с рациональными вершинами.}
    \end{align*}

    ~\\

    2) Введём обозначение: $\mathcal{A} = \sigma\left(\Box\right)$ --- $\sigma$-алгебра, порождённая всеми
    открытыми прямоугольниками. Известно, что $\mathcal{A} = \mathcal{B}(\RR^2)$.
    По определению $\mathcal{B}(\RR) \otimes \mathcal{B}(\RR) =
    \sigma\left( \left\{ B_1 \times B_2 \colon B_i \in \mathcal{B}(\RR) \right\} \right)$. Множество
    $B_1 \times B_{2}, B_i \in \mathcal{B}(\RR)$ называется \it{измеримым прямоугольником}.
    Для большего понимания происходящего изобразим на плоскости открытый прямоугольник и измеримый
    прямоугольник:
    \begin{align*}
        &\begin{tikzpicture}[line cap=round,line join=round,>=triangle 45,x=1.4,y=1.5]
             \draw[->,color=black] (-1.98,0) -- (141.21,0);
             \foreach \x in {,10,20,30,40,50,60,70,80,90,100,110,120,130,140}
             \draw[shift={(\x,0)},color=black] (0pt,-2pt);
             \draw[->,color=black] (0,-1.54) -- (0,114.35);
             \foreach \y in {,10,20,30,40,50,60,70,80,90,100,110}
             \draw[shift={(0,\y)},color=black] (-2pt,0pt);
             \clip(-1.98,-1.54) rectangle (141.21,114.35);
             \fill[dash pattern=on 7pt off 7pt,color=ttttff,fill=ttttff,fill opacity=0.2] (6.45,81.86) -- (15.84,81.86) -- (15.84,95.3) -- (6.45,95.3) -- cycle;
             \fill[dash pattern=on 7pt off 7pt,color=ttttff,fill=ttttff,fill opacity=0.2] (30,81.86) -- (40,81.86) -- (40,95.3) -- (30,95.3) -- cycle;
             \fill[dash pattern=on 7pt off 7pt,color=ttttff,fill=ttttff,fill opacity=0.2] (48.52,81.86) -- (50,81.86) -- (50,95.3) -- (48.52,95.3) -- cycle;
             \fill[dash pattern=on 7pt off 7pt,color=ttttff,fill=ttttff,fill opacity=0.2] (60,81.86) -- (90,81.86) -- (90,95.3) -- (60,95.3) -- cycle;
             \fill[dash pattern=on 7pt off 7pt,color=ttttff,fill=ttttff,fill opacity=0.2] (105.89,81.86) -- (114.35,81.86) -- (114.35,95.3) -- (105.89,95.3) -- cycle;
             \fill[dash pattern=on 7pt off 7pt,color=ttttff,fill=ttttff,fill opacity=0.2] (105.89,64.24) -- (105.89,61.34) -- (114.35,61.34) -- (114.35,64.24) -- cycle;
             \fill[dash pattern=on 7pt off 7pt,color=ttttff,fill=ttttff,fill opacity=0.2] (105.89,40) -- (114.35,40) -- (114.35,22.87) -- (105.89,22.87) -- cycle;
             \fill[dash pattern=on 7pt off 7pt,color=ttttff,fill=ttttff,fill opacity=0.2] (105.89,16.96) -- (114.35,16.96) -- (114.35,10) -- (105.89,10) -- cycle;
             \fill[dash pattern=on 7pt off 7pt,color=ttttff,fill=ttttff,fill opacity=0.2] (90,10) -- (90,16.96) -- (60,16.96) -- (60,10) -- cycle;
             \fill[dash pattern=on 7pt off 7pt,color=ttttff,fill=ttttff,fill opacity=0.2] (50,10) -- (50,16.96) -- (48.52,16.96) -- (48.52,10) -- cycle;
             \fill[dash pattern=on 7pt off 7pt,color=ttttff,fill=ttttff,fill opacity=0.2] (40,16.96) -- (40,10) -- (30,10) -- (30,16.96) -- cycle;
             \fill[dash pattern=on 7pt off 7pt,color=ttttff,fill=ttttff,fill opacity=0.2] (15.84,10) -- (15.84,16.96) -- (6.45,16.96) -- (6.45,10) -- cycle;
             \fill[dash pattern=on 7pt off 7pt,color=ttttff,fill=ttttff,fill opacity=0.2] (15.84,22.87) -- (15.84,40) -- (6.45,40) -- (6.45,22.87) -- cycle;
             \fill[dash pattern=on 7pt off 7pt,color=ttttff,fill=ttttff,fill opacity=0.2] (6.45,61.34) -- (6.45,64.24) -- (15.84,64.24) -- (15.84,61.34) -- cycle;
             \fill[dash pattern=on 7pt off 7pt,color=ttttff,fill=ttttff,fill opacity=0.2] (60,64.24) -- (90,64.24) -- (90,61.34) -- (60,61.34) -- cycle;
             \fill[dash pattern=on 7pt off 7pt,color=ttttff,fill=ttttff,fill opacity=0.2] (60,40) -- (90,40) -- (90,22.87) -- (60,22.87) -- cycle;
             \fill[dash pattern=on 7pt off 7pt,color=ttttff,fill=ttttff,fill opacity=0.2] (50,22.87) -- (50,40) -- (48.52,40) -- (48.52,22.87) -- cycle;
             \fill[dash pattern=on 7pt off 7pt,color=ttttff,fill=ttttff,fill opacity=0.2] (40,22.87) -- (30,22.87) -- (30,40) -- (40,40) -- cycle;
             \fill[dash pattern=on 7pt off 7pt,color=ttttff,fill=ttttff,fill opacity=0.2] (30,61.34) -- (30,64.24) -- (40,64.24) -- (40,61.34) -- cycle;
             \fill[dash pattern=on 7pt off 7pt,color=ttttff,fill=ttttff,fill opacity=0.2] (48.52,61.34) -- (48.52,64.24) -- (50,64.24) -- (50,61.34) -- cycle;
             \draw [line width=2.8pt,color=ttttff] (0,10)-- (0,16.96);
             \draw [line width=2.8pt,color=qqqqff] (0,22.87)-- (0,40);
             \draw [line width=2.8pt,color=ttttff] (0,61.34)-- (0,64.24);
             \draw [line width=2.8pt,color=ttttff] (0,81.86)-- (0,95.3);
             \draw [line width=2.8pt,color=ttttff] (6.45,0)-- (15.84,0);
             \draw [line width=2.8pt,color=ttttff] (30,0)-- (40,0);
             \draw [line width=2.8pt,color=ttttff] (48.52,0)-- (50,0);
             \draw [line width=2.8pt,color=ttttff] (60,0)-- (90,0);
             \draw [line width=2.8pt,color=ttttff] (105.89,0)-- (114.35,0);
             \draw [dash pattern=on 7pt off 7pt,color=ttttff] (6.45,81.86)-- (15.84,81.86);
             \draw [dash pattern=on 7pt off 7pt,color=ttttff] (15.84,81.86)-- (15.84,95.3);
             \draw [dash pattern=on 7pt off 7pt,color=ttttff] (15.84,95.3)-- (6.45,95.3);
             \draw [dash pattern=on 7pt off 7pt,color=ttttff] (6.45,95.3)-- (6.45,81.86);
             \draw [dash pattern=on 7pt off 7pt,color=ttttff] (30,81.86)-- (40,81.86);
             \draw [dash pattern=on 7pt off 7pt,color=ttttff] (40,81.86)-- (40,95.3);
             \draw [dash pattern=on 7pt off 7pt,color=ttttff] (40,95.3)-- (30,95.3);
             \draw [dash pattern=on 7pt off 7pt,color=ttttff] (30,95.3)-- (30,81.86);
             \draw [dash pattern=on 7pt off 7pt,color=ttttff] (48.52,81.86)-- (50,81.86);
             \draw [dash pattern=on 7pt off 7pt,color=ttttff] (50,81.86)-- (50,95.3);
             \draw [dash pattern=on 7pt off 7pt,color=ttttff] (50,95.3)-- (48.52,95.3);
             \draw [dash pattern=on 7pt off 7pt,color=ttttff] (48.52,95.3)-- (48.52,81.86);
             \draw [dash pattern=on 7pt off 7pt,color=ttttff] (60,81.86)-- (90,81.86);
             \draw [dash pattern=on 7pt off 7pt,color=ttttff] (90,81.86)-- (90,95.3);
             \draw [dash pattern=on 7pt off 7pt,color=ttttff] (90,95.3)-- (60,95.3);
             \draw [dash pattern=on 7pt off 7pt,color=ttttff] (60,95.3)-- (60,81.86);
             \draw [dash pattern=on 7pt off 7pt,color=ttttff] (105.89,81.86)-- (114.35,81.86);
             \draw [dash pattern=on 7pt off 7pt,color=ttttff] (114.35,81.86)-- (114.35,95.3);
             \draw [dash pattern=on 7pt off 7pt,color=ttttff] (114.35,95.3)-- (105.89,95.3);
             \draw [dash pattern=on 7pt off 7pt,color=ttttff] (105.89,95.3)-- (105.89,81.86);
             \draw [line width=1.6pt,color=wwwwff] (105.89,74.21)-- (114.35,74.21);
             \draw [dash pattern=on 7pt off 7pt,color=ttttff] (105.89,64.24)-- (105.89,61.34);
             \draw [dash pattern=on 7pt off 7pt,color=ttttff] (105.89,61.34)-- (114.35,61.34);
             \draw [dash pattern=on 7pt off 7pt,color=ttttff] (114.35,61.34)-- (114.35,64.24);
             \draw [dash pattern=on 7pt off 7pt,color=ttttff] (114.35,64.24)-- (105.89,64.24);
             \draw [dash pattern=on 7pt off 7pt,color=ttttff] (105.89,40)-- (114.35,40);
             \draw [dash pattern=on 7pt off 7pt,color=ttttff] (114.35,40)-- (114.35,22.87);
             \draw [dash pattern=on 7pt off 7pt,color=ttttff] (114.35,22.87)-- (105.89,22.87);
             \draw [dash pattern=on 7pt off 7pt,color=ttttff] (105.89,22.87)-- (105.89,40);
             \draw [dash pattern=on 7pt off 7pt,color=ttttff] (105.89,16.96)-- (114.35,16.96);
             \draw [dash pattern=on 7pt off 7pt,color=ttttff] (114.35,16.96)-- (114.35,10);
             \draw [dash pattern=on 7pt off 7pt,color=ttttff] (114.35,10)-- (105.89,10);
             \draw [dash pattern=on 7pt off 7pt,color=ttttff] (105.89,10)-- (105.89,16.96);
             \draw [dash pattern=on 7pt off 7pt,color=ttttff] (90,10)-- (90,16.96);
             \draw [dash pattern=on 7pt off 7pt,color=ttttff] (90,16.96)-- (60,16.96);
             \draw [dash pattern=on 7pt off 7pt,color=ttttff] (60,16.96)-- (60,10);
             \draw [dash pattern=on 7pt off 7pt,color=ttttff] (60,10)-- (90,10);
             \draw [dash pattern=on 7pt off 7pt,color=ttttff] (50,10)-- (50,16.96);
             \draw [dash pattern=on 7pt off 7pt,color=ttttff] (50,16.96)-- (48.52,16.96);
             \draw [dash pattern=on 7pt off 7pt,color=ttttff] (48.52,16.96)-- (48.52,10);
             \draw [dash pattern=on 7pt off 7pt,color=ttttff] (48.52,10)-- (50,10);
             \draw [dash pattern=on 7pt off 7pt,color=ttttff] (40,16.96)-- (40,10);
             \draw [dash pattern=on 7pt off 7pt,color=ttttff] (40,10)-- (30,10);
             \draw [dash pattern=on 7pt off 7pt,color=ttttff] (30,10)-- (30,16.96);
             \draw [dash pattern=on 7pt off 7pt,color=ttttff] (30,16.96)-- (40,16.96);
             \draw [dash pattern=on 7pt off 7pt,color=ttttff] (15.84,10)-- (15.84,16.96);
             \draw [dash pattern=on 7pt off 7pt,color=ttttff] (15.84,16.96)-- (6.45,16.96);
             \draw [dash pattern=on 7pt off 7pt,color=ttttff] (6.45,16.96)-- (6.45,10);
             \draw [dash pattern=on 7pt off 7pt,color=ttttff] (6.45,10)-- (15.84,10);
             \draw [dash pattern=on 7pt off 7pt,color=ttttff] (15.84,22.87)-- (15.84,40);
             \draw [dash pattern=on 7pt off 7pt,color=ttttff] (15.84,40)-- (6.45,40);
             \draw [dash pattern=on 7pt off 7pt,color=ttttff] (6.45,40)-- (6.45,22.87);
             \draw [dash pattern=on 7pt off 7pt,color=ttttff] (6.45,22.87)-- (15.84,22.87);
             \draw [dash pattern=on 7pt off 7pt,color=ttttff] (6.45,61.34)-- (6.45,64.24);
             \draw [dash pattern=on 7pt off 7pt,color=ttttff] (6.45,64.24)-- (15.84,64.24);
             \draw [dash pattern=on 7pt off 7pt,color=ttttff] (15.84,64.24)-- (15.84,61.34);
             \draw [dash pattern=on 7pt off 7pt,color=ttttff] (15.84,61.34)-- (6.45,61.34);
             \draw [line width=1.6pt,color=wwwwff] (6.45,74.21)-- (15.84,74.21);
             \draw [line width=1.6pt,color=wwwwff] (30,74.21)-- (40,74.21);
             \draw [line width=1.6pt,color=wwwwff] (48.52,74.21)-- (50,74.21);
             \draw [line width=1.6pt,color=wwwwff] (60,74.21)-- (90,74.21);
             \draw [dash pattern=on 7pt off 7pt,color=ttttff] (60,64.24)-- (90,64.24);
             \draw [dash pattern=on 7pt off 7pt,color=ttttff] (90,64.24)-- (90,61.34);
             \draw [dash pattern=on 7pt off 7pt,color=ttttff] (90,61.34)-- (60,61.34);
             \draw [dash pattern=on 7pt off 7pt,color=ttttff] (60,61.34)-- (60,64.24);
             \draw [dash pattern=on 7pt off 7pt,color=ttttff] (60,40)-- (90,40);
             \draw [dash pattern=on 7pt off 7pt,color=ttttff] (90,40)-- (90,22.87);
             \draw [dash pattern=on 7pt off 7pt,color=ttttff] (90,22.87)-- (60,22.87);
             \draw [dash pattern=on 7pt off 7pt,color=ttttff] (60,22.87)-- (60,40);
             \draw [dash pattern=on 7pt off 7pt,color=ttttff] (50,22.87)-- (50,40);
             \draw [dash pattern=on 7pt off 7pt,color=ttttff] (50,40)-- (48.52,40);
             \draw [dash pattern=on 7pt off 7pt,color=ttttff] (48.52,40)-- (48.52,22.87);
             \draw [dash pattern=on 7pt off 7pt,color=ttttff] (48.52,22.87)-- (50,22.87);
             \draw [dash pattern=on 7pt off 7pt,color=ttttff] (40,22.87)-- (30,22.87);
             \draw [dash pattern=on 7pt off 7pt,color=ttttff] (30,22.87)-- (30,40);
             \draw [dash pattern=on 7pt off 7pt,color=ttttff] (30,40)-- (40,40);
             \draw [dash pattern=on 7pt off 7pt,color=ttttff] (40,40)-- (40,22.87);
             \draw [dash pattern=on 7pt off 7pt,color=ttttff] (30,61.34)-- (30,64.24);
             \draw [dash pattern=on 7pt off 7pt,color=ttttff] (30,64.24)-- (40,64.24);
             \draw [dash pattern=on 7pt off 7pt,color=ttttff] (40,64.24)-- (40,61.34);
             \draw [dash pattern=on 7pt off 7pt,color=ttttff] (40,61.34)-- (30,61.34);
             \draw [dash pattern=on 7pt off 7pt,color=ttttff] (48.52,61.34)-- (48.52,64.24);
             \draw [dash pattern=on 7pt off 7pt,color=ttttff] (48.52,64.24)-- (50,64.24);
             \draw [dash pattern=on 7pt off 7pt,color=ttttff] (50,64.24)-- (50,61.34);
             \draw [dash pattern=on 7pt off 7pt,color=ttttff] (50,61.34)-- (48.52,61.34);
             \begin{scriptsize}
                 \fill [color=qqqqff] (0,74.21) ++(-2.5pt,0 pt) -- ++(2.5pt,2.5pt)--++(2.5pt,-2.5pt)--++(-2.5pt,-2.5pt)--++(-2.5pt,2.5pt);
             \end{scriptsize}
        \end{tikzpicture}
        &\begin{tikzpicture}[line cap=round,line join=round,>=triangle 45,x=18,y=18]
             \draw[->,color=black] (-0.42,0) -- (11.78,0);
             \foreach \x in {,1,2,3,4,5,6,7,8,9,10,11}
             \draw[shift={(\x,0)},color=black] (0pt,-2pt);
             \draw[->,color=black] (0,-0.29) -- (0,9.58);
             \foreach \y in {,1,2,3,4,5,6,7,8,9}
             \draw[shift={(0,\y)},color=black] (2pt,0pt) -- (-2pt,0pt);
             \clip(-0.42,-0.29) rectangle (11.78,9.58);
             \fill[dash pattern=on 3pt off 3pt,color=qqqqzz,fill=qqqqzz,fill opacity=0.1] (2.46,8.6) -- (7.02,8.6) -- (7.02,2.38) -- (2.46,2.38) -- cycle;
             \draw [line width=2.8pt,color=qqqqcc] (2.46,0)-- (7.02,0);
             \draw [line width=2.8pt,color=qqqqcc] (0,2.38)-- (0,8.6);
             \draw [dash pattern=on 3pt off 3pt,color=qqqqzz] (2.46,8.6)-- (7.02,8.6);
             \draw [dash pattern=on 3pt off 3pt,color=qqqqzz] (7.02,8.6)-- (7.02,2.38);
             \draw [dash pattern=on 3pt off 3pt,color=qqqqzz] (7.02,2.38)-- (2.46,2.38);
             \draw [dash pattern=on 3pt off 3pt,color=qqqqzz] (2.46,2.38)-- (2.46,8.6);
        \end{tikzpicture}
        \\
        &a)~ \text{Измеримый прямоугольник}
        &b)~ \text{Открытый прямоугольник}
    \end{align*}
    Очевидно, что открытый прямоугольник, как декартово произведение промежутков, является частным
    случаем измеримого прямоугольника, а потому справедливо включение
    $\mathcal{A} \subset \sigma\left( \left\{ B_1 \times B_2 \colon B_i \in \mathcal{B}(\RR) \right\} \right)$. Покажем теперь
    включение в другую сторону:

    Достаточно доказать, что каждый измеримый прямоугольник $B_1 \times B_2 \in \mathcal{A}$. В этом
    случае вместе с ними, очевидно, там будет лежать и наименьшая порождаемая ими $\sigma$-алгебра.
    Пусть $I$ --- промежуток, $B \in \mathcal{B}(\RR)$. Докажем, что $B \times I \in \mathcal{A}$.
    Рассмотрим следующий набор подмножеств $\RR$:
    $\mathcal{F} = \{ X \subseteq \RR \colon X \times I \in \mathcal{A}\}$.
    \begin{itemize}
        \item Всякий промежуток принадлежит $\mathcal{F}$ (Возьмём $X$ равный некоторому промежутку,
        тогда $X \times I$ задаёт некоторый прямоугольник, и, следовательно, лежит в $\mathcal{A}$).
        \item $\mathcal{F}$ --- $\sigma$-алгебра. Очевидно, что $\nothing \in \mathcal{F}$.
        $\RR \times I$ так же задаёт некоторый прямоугольник $\implies \RR \in \mathcal{F}$.
        Пусть теперь $X_1, X_2 \in \mathcal{F}$. Тогда $X_1 \times I$ и $X_2 \times I$ задают два
        прямоугольника, и их пересечение, объединение и разность так же являются прямоугольником
        (см. рисунок ниже), а значит, лежат в $\mathcal{F}$. (Кроме того верно, что
        $(X_1 \times I) \cap (X_2 \times I) = (X_1 \cap X_2) \times I$. Для остальных операций аналогично).
        Кроме того очеивдно, что если взять счётный набор $\{X\}_i \subseteq \mathcal{F}$, то
        для каждого $i ~ X_i \times I$ будет задавать прямоугольник, и их счётное объединение и пересечение
        также будет прямоугольником, а значит, будет лежать в $\mathcal{F}$.
    \end{itemize}
    \begin{align*}
        \definecolor{qqqqcc}{rgb}{0,0,0.8}
        \definecolor{qqwwtt}{rgb}{0,0.4,0.2}
        \definecolor{qqwwqq}{rgb}{0,0.4,0}
        \definecolor{qqqqff}{rgb}{0,0,1}
        \begin{tikzpicture}[line cap=round,line join=round,>=triangle 45,x=22,y=22]
            \draw[->,color=black] (-0.71,0) -- (10.97,0);
            \foreach \x in {,1,2,3,4,5,6,7,8,9,10}
            \draw[shift={(\x,0)},color=black] (0pt,-2pt);
            \draw[->,color=black] (0,-0.56) -- (0,8.94);
            \foreach \y in {,1,2,3,4,5,6,7,8}
            \draw[shift={(0,\y)},color=black] (2pt,0pt) -- (-2pt,0pt);
            \clip(-0.71,-0.56) rectangle (10.97,8.94);
            \fill[dash pattern=on 1pt off 1pt,pattern color=qqqqff,fill=qqqqff,pattern=north east lines] (1.68,1.68) -- (5,1.68) -- (5,3.72) -- (1.68,3.72) -- cycle;
            \fill[dash pattern=on 1pt off 1pt,pattern color=qqwwtt,fill=qqwwtt,pattern=north west lines] (3.78,1.68) -- (3.78,3.72) -- (9,3.72) -- (9,1.68) -- cycle;
            \draw [line width=2.8pt] (0,1.68)-- (0,3.72);
            \draw [line width=1.6pt,color=qqqqff] (1.68,0)-- (5,0);
            \draw [line width=1.6pt,color=qqwwqq] (3.78,0)-- (9,0);
            \draw [dash pattern=on 1pt off 1pt,color=qqqqff] (1.68,1.68)-- (5,1.68);
            \draw [dash pattern=on 1pt off 1pt,color=qqqqff] (5,1.68)-- (5,3.72);
            \draw [dash pattern=on 1pt off 1pt,color=qqqqff] (5,3.72)-- (1.68,3.72);
            \draw [dash pattern=on 1pt off 1pt,color=qqqqff] (1.68,3.72)-- (1.68,1.68);
            \draw [dash pattern=on 1pt off 1pt,color=qqwwtt] (3.78,1.68)-- (3.78,3.72);
            \draw [dash pattern=on 1pt off 1pt,color=qqwwtt] (3.78,3.72)-- (9,3.72);
            \draw [dash pattern=on 1pt off 1pt,color=qqwwtt] (9,3.72)-- (9,1.68);
            \draw [dash pattern=on 1pt off 1pt,color=qqwwtt] (9,1.68)-- (3.78,1.68);
            \draw (3.24,4.43) node[anchor=north west] {$(X_1 \cap X_2) \times I$};
            \draw [color=qqqqcc](2.65,1.6) node[anchor=north west] {\mathbf{$ X_1 \times I $}};
            \draw [color=qqwwqq](6,1.6) node[anchor=north west] {\mathbf{$X_2 \times I$}};
            \draw (-0.63,3) node[anchor=north west] {\mathbf{$ I $}};
            \draw [color=qqqqff](2.95,0) node[anchor=north west] {\mathbf{$ X_1 $}};
            \draw [color=qqwwqq](5.83,-0.03) node[anchor=north west] {\mathbf{$ X_2 $}};
            \draw [color=qqqqcc](4.74,0.38) node[anchor=north west] {\mathbf{$ ) $}};
            \draw [color=qqqqcc](1.44,0.38) node[anchor=north west] {\mathbf{$ ( $}};
            \draw [color=qqwwqq](3.54,0.38) node[anchor=north west] {\mathbf{$ ( $}};
            \draw [color=qqwwqq](8.74,0.38) node[anchor=north west] {\mathbf{$ ) $}};
        \end{tikzpicture}
    \end{align*}
    Отсюда следует, что $\mathcal{B}(\RR) \subset \mathcal{F}$. Рассмотрим теперь следующий набор подмножеств
    в $\RR$:
    $\widetilde{\mathcal{F}} = \{X \subseteq \RR \colon B \times X \in \mathcal{A}$\}.
    \begin{enumerate}
        \item $\widetilde{\mathcal{F}}$ содержит все промежутки. Действительно, положим $X = I$ --- некоторому
        промежутку. Тогда необходимо доказать, что $B \times I \in A$. Но известно, что
        $\mathcal{B}(\RR) \subset \mathcal{F}$, а значит $B \times I \in A ~ \forall B \in \mathcal{B}(\RR)$.
        \item $\widetilde{\mathcal{F}}$ --- $\sigma$-алгебра. Это вобщем-то очевидно, т.к. все действия, которые
        нужно проверить, мы будем делать со второй компонентой, а для неё уже всё доказано.
    \end{enumerate}
    Отсюда следует, что $\mathcal{B}(\RR) \subset \widetilde{\mathcal{F}}$. Таким образом мы доказали
    требуемое, а именно, что каждый измеримый прямоугольник $B_1 \times B_2$ лежит в $\mathcal{A}$, а
    значит, $\mathcal{A} \supset \sigma\left( \left\{ B_1 \times B_2 \colon B_i \in \mathcal{B}(\RR) \right\} \right)$.

    Показали включение в обе стороны, а значит
    \[
        \mathcal{B}(\RR^2) = \sigma(\Box) = \mathcal{A} =
        \sigma\left( \left\{ B_1 \times B_2 \colon B_i \in \mathcal{B}(\RR) \right\} \right)
        = \mathcal{B}(\RR) \otimes \mathcal{B}(\RR)
    \]
\end{problem}
\begin{exercise}
    Вернёмся к первому пункту предыдущей задачи: известно, что в $U$ содержится континуум точек.
    При этом для каждой точки мы вписываем прямоугольник открытый шар этой точки. Получается,
    что прямоугольников будет столько же, сколько точек. Почему же мы тогда утверждаем,
    что прямоугольников будет не более чем счётно?
\end{exercise}