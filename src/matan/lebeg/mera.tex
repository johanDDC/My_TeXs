\section{Мера.}
Пусть $\mathcal{A}$ --- произвольная алгебра.
\begin{definition}
    Функция $\mu \colon \mathcal{A} \to [0, +\infty)$ называется \it{мерой}, если она обладает
    свойством аддитивности:
    \[
        \mu\left( \bigsqcup_{n = 1}^N A_n \right) = \sum\limits_{n = 1}^N \mu(A_n), \:\:\: A_n \in \mathcal{A}
    \]
    Мера $\mu$ называется \it{счётно-аддитивной}, если для неё выполнена \it{счётная аддитивность}:
    \begin{align*}
        \forall \: A_n \in \mathcal{A} \colon \bigsqcup_{n = 1}^{\infty} A_n \in \mathcal{A}
        \implies
        \mu\left( \bigsqcup_{n = 1}^{\infty} A_n \right) = \sum\limits_{n = 1}^{\infty} \mu(A_n)
    \end{align*}
\end{definition}
\begin{example}{\ }

    Пусть $\Omega$ --- вероятностное пространство, и $|\Omega| < \infty$. Тогда $2^{\Omega}$ очевидно
    алгебра. Тогда $P \colon 2^{\Omega} \to [0, 1] \subset [0, +\infty)$ --- вероятностная мера, по
    определению является мерой.

    Пусть теперь $\Omega$ --- счётное вероятностное пространство. Тогда вероятностная мера на нём является
    счётно-аддитивной мерой.
\end{example}
\begin{comment}
    \hyperref[length example]{В пункте 2.2 мы привели пример алгебры} $\mathcal{A}$ конечных объединений
    попарно непересекающихся промежутков на отрезке $[0, 1]$. На этом отрезке есть мера $\lambda$,
    которую мы привыкли понимать как ''длина промежутков``. Нам хочется знать, а можно ли такую удобную
    функцию продолжить с такой области определения на нечто большее. Например на алгебру, порождаемую
    этой. Нетрудно видеть, что $\sigma(\mathcal{A}) = \mathcal{B}[0, 1]$. Таким образом если продолжим
    $\lambda$ на Борелевскую алгебру, мы сможем находить длину (далее будем всё же использовать термин
    ''мера``) практически произвольного подмножества в $[0, 1]$, а значит, на самом деле, в любом отрезке.
\end{comment}
\begin{example}
    Пусть $X = \NN$, и $\mathcal{A} = \{$ конечные множества и дополнения к ним $\}$ --- алгебра
    (доказывалось в \hyperref[alg_example1]{этом примере}). Введём на этой алгебре меру:
    \[
        \mu(B) =
        \begin{cases}
            0 & |B| < \infty \text{ ($B$ --- конечное множество)}\\
            1 & |B| = |\NN|  \text{ ($B$ --- дополнение к конечному множеству)}
        \end{cases}
    \]
    В одну сторону очевидно, что $\mu$ --- аддитивная мера. В другую сторону тривиально, что
    $\mu$ не является счётно-аддитивной мерой: действительно, возьмём одноэлементные
    множества $B_i$ --- $i$-ое чётное число. Тогда если если $\mu$ --- счётно-аддитивна, то
    мера их счётного объединения равна сумме мер каждого множества, но мера каждого отдельного
    $B_i$ равно $0$. С другой стороны очевидно, что счётное объединение $B_i$ является счётным
    множеством всех чётных чисел ($2\NN$), а значит его мера должна быть $1$. Противоречие.
\end{example}
\begin{proposal}
    Пусть $\mu$ --- аддитивная мера на алгебре $\mathcal{A}$. Тогда $\mu$ --- $\sigma$-аддитивна
    $\iff \mathcal{A} \ni A_1 \supset A_2 \supset A_3 \supset \ldots$, таких что
    $\bigcap_{n = 1}^{\infty} A_n = \nothing \implies \lim\mu(A_n) = 0$. Т.е. $\mu$ счётно-аддитивна
    тогда и только тогда, когда для всякой последовательности вложенных множеств, таких что их пересечение
    пусто, предел $\lim\mu(A_n) = 0$.
\end{proposal}
\begin{proof}
    ~\\
    $\Rightarrow \colon$ Пусть $C_i = A_i \setminus A_{i+1}$ --- $i$-ый слой. Все слои попарно
    не пересекаются. Тогда
    \[
        A_n = \bigcup_{k = n}^{\infty} C_k \implies
        \mu(A_n) = \sum\limits_{k = n}^{\infty} \mu(C_k)
    \]
    Т.к. мера $\mu$ --- счётно-аддитивна, то ряд $\mu(A_1) = \sum\limits_{k = 1}^{\infty} \mu(C_k)$
    сходится. Тогда $\mu(A_n) = r_n$ остаток этого ряда. Т.к. ряд сходится, то $r_n \to 0 \iff
    \mu(A_n) \to 0$.

    $\Leftarrow \colon$ Пусть $\{C\}_n \subset \mathcal{A}$ --- набор попарно непересекающихся множеств,
    и известно, что $\bigcup_{k = 1}^{\infty} = A_1 \in \mathcal{A}$. Пусть $A_{n + 1} =
    \bigcup_{k = n + 1}^{\infty} C_k$, тогда $A_{n + 1} \subset A_n$ и $\bigcap_{k = 1}^{\infty}
    A_k = \nothing$. По конечной аддитивности: $\mu\left( \bigcup_{k=1}^{N} C_k \right)
    =\sum\limits_{k=1}^{N} \mu(C_k)$. Тогда
    \[
        \mu\left( \bigcup_{k=1}^{\infty} C_k \right) -
        \mu\left( \bigcup_{k=1}^{n} C_k \right) = \mu(A_{n+1}) \to 0
    \]
    Отсюда следует счётная аддитивность.
\end{proof}
\subsection{Внешняя мера.}
Пусть $X$ --- некоторое множество, $\mathcal{A}$ --- алгебра на $X$, $\mu$ --- мера на $\mathcal{A}$.
Мы научились измерять произвольные наборы множеств, лежащие в алгебре. Теперь мы хоти уметь измерить произвольное
подмножество в $X$.\\
Пусть $E \subset X$ --- произвольное подмножество.
\begin{definition}
    Число $\mu^*(E)$ называется \it{внешней мерой $E$} и определяется следующим образом:
    \[
        \mu^*(E) = \inf\left\{ \sum\limits_{k = 1}^{\infty} \mu(A_k) \colon A_k \in \mathcal{A}, ~
        E \subset \bigcup_{k = 1}^{\infty} A_k\right\}
    \]
\end{definition}
Т.е. мы берём не более чем счётный набор множеств из алгебры, такой что в объединении он накрывает $E$, и тогда внешняя мера
$E$ это самая точная нижняя оценка на сумму мер такого набора. Понятно, что всегда можно взять само множество $X$: мы знаем, что
оно всегда лежит в алгебре и оно точно накроет $E$. Однако почти никогда мера всего множества адекватно не приблизит меру его произвольного
подмножества.
\begin{comment}
    Вообще говоря \textbf{внешняя мера не является мерой}.
\end{comment}
\begin{example}
    Продолжим мучить наш пример:
    пусть $X = \NN$, $\mathcal{A} = \{\text{конечные множества и дополнения к ним}\}$, и мера $\mu$ на $\mathcal{A}$
    определяется как $\mu(A) = 0$ если $A$ конечное, и $\mu(A) = 1$, если $A$ --- дополнение к конечному. Посмотрим как
    выглядит внешняя мера на этом примере: пусть $E_1$ --- все чётные числа, $E_2$ --- все нечётные числа. Тогда
    \begin{align*}
        &\mu^*(E_1) = 0\\
        &\mu^*(E_2) = 0\\
        &\mu^*(\NN) = 0
    \end{align*}
    Каждое чётное число лежит в алгебре, и мера каждого такого числа равна $0$, отсюда значение внешней меры на чётных числах.
    Аналогично для нечётных. И аналогично для всего $X$: каждое натуральное число лежит в $\mathcal{A}$, и мера каждого
    натурального числа равна $0$. Таким образом самая точная оценка на сумму их мер равна $0$. Пример интересен тем, что
    внешняя мера на таком множестве с такой алгеброй и мерой получилась просто тождественным нулём ($\mu^* \equiv 0$).
    И вот этот новый, придуманный нами способ измерить любое подмножество не имеет ничего общего с тем, как мы меряли множества
    раньше, т.к. $0 = \mu^*(\NN) \neq \mu(\NN) = 1$.
\end{example}
\begin{example}
    Пусть $X = \{2, 3, 4\}$, $\mathcal{A} = \{\nothing, X\}, \mu(\nothing) = 0, \mu(X) = 1$. Посмотрим на внешнюю меру на таком
    примере:
    \begin{align*}
        &\mu^*(\nothing) = 0 & \mu^*(\{2\}) = \mu^*(\{3\}) = \mu^*(\{4\}) = \mu^*(X) = 1
    \end{align*}
    В этом примере внешняя мера даже не обладает свойством аддитивности
    ($\mu^*(\{2\}) + \mu^*(\{3\}) = 2 \neq 1 = \mu^*(\{2, 3\})$), а значит не является мерой в нашем понимании. Понятно,
    что имея такую скудную алгебру покрыть любое непустое множество можно только множеством $X$.
\end{example}
Из этих двух примеров мы можем сформулировать следующее желание: мы хотим получить такую функцию, которая на известных
множествах из алгебры совпадала бы с их уже определённой мерой, а для всех остальных множеств являлась бы хотя бы аддитивной
(т.е. являлась хотя бы мерой в нашем понимании), а ещё лучше: счётно-аддитивной.
\subsection{Теорема Лебега.}
Пусть $X$ --- произвольное множество, $\mathcal{A}$ --- алгебра на $X$, $\mu$ --- мера на $\mathcal{A}$.
\begin{definition}
    Множество $E \subset X$ называется \it{измеримым}, если принадлежит \it{классу измеримых множеств} $\mathcal{A}_{\mu}$,
    который определяется следующим образом:
    \[
        \mathcal{A}_{\mu} = \left\{ E \subset X \colon \forall~\epsilon > 0 ~\exists~A_{\epsilon} \in \mathcal{A}
        \implies \mu^*(E \vartriangle A_{\epsilon}) < \epsilon \right\}
    \]
    где $\vartriangle$ --- симметрическая разность множеств.\\
    Можно понимать это так: как только мы максимально покрыли наше множество $E$ множествами из алгебры, может случиться
    такое, что остался некоторый зазор, свободный участок множества $E$, не покрываемый ни одним из множеств $A_i \in \mathcal{A}$.
    Симметрическая разность множеств это всё равно что взять их объединене и выкинуть их пересечение. А мера симметрической разности
    можества $E$, и набора, который его покрывает, это как раз мера того самого зазора, который мы не покрыли, и чтобы множество
    было измеримо, мы требуем, чтобы мера этого зазора была сколь угодно мала.
\end{definition}
\begin{theorem}[Лебега]
    Пусть $\mu$ --- счётно-аддитивная мера на алгебре $\mathcal{A}$. Тогда
    \begin{enumerate}
        \item $\mathcal{A}_{\mu}$ --- $\sigma$-алгебра.
        \item $\mathcal{A}_{\mu} \supset \sigma(\mathcal{A})$.
        \item $\mu^*$ --- счётно-аддитивная мера на $\mathcal{A}_{\mu}$.
        \item $\mu^* = \mu$ на $\mathcal{A}$.
        \item $\mu^*$ --- единственное продолжение $\mu$ на $\sigma(\mathcal{A})$ и на $\mathcal{A}_{\mu}$.
    \end{enumerate}
\end{theorem}
\begin{proof}
    Не будет приводиться.
\end{proof}
Таким образом мы получили самую важную теорему, которая одновременно даёт нам ответы на все вопросы, которые только можно
было задать про измерения ''длины множеств``. Самое сложное действие, которое нам надо проделать, чтобы научиться измерять
какое-то множество, это проверить, мера на данной алгебре действительно является счётно-аддитивной.