\section{Пространные и неформальные рассуждения о понятии длины множества.}

Мы считаем, что найти длину произвольного отрезка или полуинтервала на $\RR$ впринцыпе задача тривиальная:
например если $[a, b) \in \RR$, то длина такого полуинтервала равна $b - a$. При этом мы считаем, что у нас есть
некоторое множество ($[0, 1)$) эталонной длины $1$, и длиной всех остальных полуинтервалов мы называем
такое число, которое обозначает, сколько раз эталонное множество вошло в данный полуинтервал. При этом нас не
смущает, если ответ получается не целым, а даже иррациональным. В самом деле, мы без проблем поверим, что длина
полуинтервала $[0, \sqrt{2})$ равна $\sqrt{2}$, т.е. эталонный полуинтервал $[0, 1)$ входит в данный полуинтервал
ровно $\sqrt{2}$ раз, что бы это ни значило.

Если мы имеем дело с двумерным множеством, то мы даже ввели специальное слово, чтобы научиться измерять его \it{длину}:
\it{площадь}. Однако такое понятие мы уже определяем с трудом. Мы научились считать площадь каких-то простых фигур,
и ввели отдельное понятие для множества, ассоциированного с конкретной функцией --- \it{криволинейная трапеция}.

И чем выше размерность, тем грустнее нам становится считать различные представления длин множеств:
\it{площади}, \it{объёмы}, \it{$k$-мерные объёмы}. Однако проблемы могут возникнуть даже в одномерном случае:

Пусть $E \subset \RR$ --- произвольное подмножество в $\RR$. Как тогда посчитать его длину в общем случае?
Мы умеем находить длину поуинтервалов, ну тогда для нас не является проблемой найти длину конечного набора
полуинтервалов. Что если множество $E$ не представимо, как конечный набор полуинтервалов? Применяя теорию рядов
мы можем посчитать длинну счётного набора полуинтервалов. Однако что если $E$ не представимо даже счётным набором?
Хорошей идеей может быть попытаться приблизить множество $E$ каким-то другим множеством, или набором множеств,
длины которых мы умеем считать. Однако сразу возникает вопрос: что значит \it{приблизить множество}, и что
значит, что \it{множества будут мало отличаться}? Приведём простейший пример:\\
Пусть нашим множеством является точка $a \in \RR$
\begin{center}
    \begin{tikzpicture}[>=latex]
        \draw[->] (0, 0) -- (14, 0);
        \fill[red] (7,0) circle (2pt);
        \node at (7, 0.3) {\color{red} $a$};
        \node at (5.5, 0) {\color{red} $\Big($};
        \node at (5.5, 0.5) {\color{red} $a - \epsilon$};
        \node at (8.5, 0) {\color{red} $\Big)$};
        \node at (8.5, 0.5) {\color{red} $a + \epsilon$};
        \node at (14, -0.3) {$\RR$};
    \end{tikzpicture}
\end{center}
Мы знаем, что длина точки равна $0$. Однако это совершенно не мешает нам пытаться приблизить её некоторым интервалом
$(a - \epsilon, a + \epsilon)$. Но совершенно не очевидно, что значит \it{интервал приближает точку}. В каком смысле
он её \it{приближает}? Он содержит ещё несчётное множество других точек, отличных от $a$, да и длину его мы можем посчитать,
и она будет зависеть от $\epsilon$ и быть больше $0$. Это у нас с точкой такие проблемы возникли, а что же будет,
когда мы начнём говорить вообще о произвольном множестве?

Мы пока не можем ответить на вопрос ''что такое длина произвольного множества``, но мы можем попытаться сформулировать
свойства, которые, мы ожидаем, будут выполняться. Пусть $l$ --- длина множества, тогда
\begin{itemize}
    \item $\forall\: A \subset \RR^n$ выполнено $l(A) \geq 0$, в частности длина не может быть отрицательной.
    \item Если $A \cap B = \nothing$, то $l(A \cup B) = l(A) + l(B)$ (длина аддитивна).
    \item Длина уважает движение, в частности:
    \begin{itemize}
        \item[\circ] Если $U$ --- произвольное ортогональное преобразование, и $U(A) = B$, то $l(U(A)) = l(B)$.
        \item[\circ] Если $h$ --- произвольное движение, и $A + h = B$, то $l(A + h) = l(B)$
    \end{itemize}
    т.е. если мы взяли некоторое множество и повертели его, то его длина не изменилась. Аналогично если
    мы взяли множество, и передвинули все его элементы на одно и то же значение, его длина не изменилась.
    \item $l([0, 1]^n) = 1$, в частности длина единичного куба любой размерности всегда равна $1$.
\end{itemize}
В дальнейшем используя словосочетание ''длина множества`` я буду иметь ввиду длину в приведённом смысле.

Итак, на данном этапе мы можем сформулировать два вопроса, на которые в дальнейшем будем исккать ответ:
\begin{enumerate}
    \item Чем приблизить произвольное множество? А если мы знаем чем, то как именно надо приближать?
    \item А возможно ли вообще измерить длину произвольного множества?
\end{enumerate}
Прямо сейчас приведём пример, который даст нам ответ на второй вопрос:

\subsection{Множество Витали}
Рассмотрим отрезок $[0, 1]$, и введём на нём отношение эквивалентности:
будем считать, что $x \sim y \iff x - y \in \QQ$. Т.е. $x$ эквивалентно $y$, если разница между ними является
рациональным числом, или, что то же самое: если одно получается из другого путём сдвига на рациональное число.
Можно строго проверить, что данное отношение действительно является отношением эквивалентно, но это тривиально
следует из коммутативности и ассоциативности операции сложения в $\QQ$.\\
Тогда весь отрезок разбивается на классы эквивалентности. Выберем из каждого класса по представителю, и сложим
их в одно множество $V$: $V = \{x_i \:|\: x_i \in [x_i]\}$.
\begin{proposal}
    Для множества $V$ мы не можем посчитать длину.
\end{proposal}
\begin{proof}
    Рассмотрим все рациональные сдвиги множества $V + r_n$ из отрезка $r_n \in \QQ \cap [-1, 1]$. Т.е мы
    рассматриваем последовательность множеств следующего строения:
    \[
        V_n = V + r_n = \left\{x_i + r_n \:|\: x_i \in V,\, r_n \in \QQ \cap [-1, 1]\right\}
    \]
    Понятно, что $\forall\: n\; V_n \subset [-1, 2]$. С другой стороны очевидно, что объединение всех $V_n$
    содержит исходный отрезок $[0, 1]$. Итого имеем следующее двойное включение:
    \[
        [0, 1] \subset \bigcup_n V_n \subset [-1, 2]
    \]
    Заметим, что $\forall\: n \neq m \implies V_n \cap V_m = \nothing$. Это тривиально проверяется, если вспомнить,
    что в $V$ содержится ровно по одному представителю из каждого класса. Естесственным свойством длинны является аддитивность,
    т.е.
    \[
        l\left( \bigcup_n V_n \right) = \sum\limits_{n} l(V_n)
    \]
    В самом деле, звучит логично, что если мы имеем множество, составленное из непересекающихся отрезков,
    то длину этого множества мы можем найти как сумму длин всех входящих в него отрезков.\\
    Известно, что $l([0, 1]) = 1; \: l([-1, 2]) = 3$. Тогда имеем следующее двойное неравенство:
    \[
        l([0, 1]) \leq l\left( \bigcup_n V_n \right) \leq l([-1, 2]) \iff
        1 \leq \sum\limits_{n} l(V_n) \leq 3
    \]
    Из свойств длины мы знаем, что $\forall\: n\: l(V_n) = l( + r_n) = l(V)$, т.к. $r_n$ это просто сдвиг
    множества $V$. Рассмотрим теперь следующие два случая:\\
    \it{Случай первый:} $l(V) = 0$. Но тогда $l(V_n) = 0$ для любого $n$. Тогда $\sum\limits_{n} l(V_n) = 0 \implies$
    двойное неравенство $1 \leq 0 \leq 3$ не выполняется. Значит, этот случай невозможен.\\
    \it{Случай второй:} $l(V) > 0$. Тогда $l(V_n) > 0$ для любого $n$, и $\sum\limits_{n} l(V_n) = \infty$. Но тогда
    двойное неравенство $1 \leq \infty \leq 3$ снова не выполняется.

    Исходя из невозможности этих двух случаев делаем вывод, что посчитать длину множества $V$ невозможно.
\end{proof}

Итак, мы смогли ответить на один из заявленных вопросов: теперь нам доподлинно известно, что для произвольного
множества измерить длину не получится, более того можно привести явный пример множества, длину которого впринцыпе
измерить нельзя. Но тогда мы, встречая новое множество даже не знаем, можем ли мы посчитать его длину или нет.
И тогда к вопросам, заявленным выше (\it{чем приближать; как приближать}) добавляется ещё один, не менее важный
вопрос: \it{что именно нужно приближать?}