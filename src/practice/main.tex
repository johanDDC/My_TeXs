\documentclass[a4paper, 12pt]{article}

%\usepackage[default]{./fonts/droidserif}
%\usepackage[defaultsans]{./fonts/droidsans}
\usepackage{../header}
\usepackage{tensor_env}

\usepackage{relsize}
\usepackage{systeme}

\geometry{a4paper,top=2cm,bottom=2cm,left=1cm,right=1cm}

\usepackage{titlepage}
\bibliographystyle{plain}

% Здесь задаем параметры титульной страницы
\setTitle{Реализация методов римановой оптимизации на многообразии тензорных поездов}
\setGroup{195}
%сюда можно воткнуть картинку подписи
\setStudentSgn{
%    \includegraphics[scale=0.1]{./sign.pdf}
}
\setStudent{И.Д. Пешехонов}
\setStudentDate{29.05.2021}
\setAdvisor{Рахуба Максим Владимирович}
\setAdvisorTitle{Доцент департамента больших данных и информационного поиска}
\setAdvisorAffiliation{Факультет компьютерных наук}
\setAdvisorDate{29.05.2021}
\setGrade{}
%сюда можно воткнуть картинку подписи
\setAdvisorSgn{
%    \includegraphics[scale=0.35]{./dad_sign.png}
}
\setYear{2021}

\title{Основы тензорноого анализа.}
\author{Пешехонов Иван. БПМИ195}

\begin{document}
    \maketitle
%    \makeTitlePage
    \pagestyle{empty}
    \tableofcontents
    \newpage
    \pagestyle{fancy}


    \section{Основные термины и обозначения}
    \begin{definition}
        \textit{$d$-мерным тензором} будем называть массив, каждый элемент которого имеет $d$ индексов. Обозначать тензоры c элементами из множества $\RR$
        будем следующим образом:
        \[
            \mbf{T} \in \RR^{n_1 \times \cdots \times n_d}
        \]
        Где число $d \in \NN$ называется \textit{порядком} тензора, а числа $n_i \in \NN \A i = 1,\ldots, d$ называются \it{модами}.
    \end{definition}\
    \begin{example}
        Тензоры порядка $1, 2$ и $3$ соответственно:
        \begin{align*}
            &\begin{pmatrix}
                 1\\2\\3\\4
            \end{pmatrix}
            &&\begin{pmatrix}
                  1 & 0 & 0 & 1\\
                  1 & 1 & 1 & 0\\
                  0 & 1 & 1 & 1\\
                  0 & 1 & 0 & 0
            \end{pmatrix}
            &&
            \left[
            \begin{pmatrix}
                1 & 0 & 1\\
                1 & 1 & 1\\
            \end{pmatrix},
            \begin{pmatrix}
                1 & 2 & 2\\
                0 & 3 & 1\\
            \end{pmatrix}
            \right]
        \end{align*}
    \end{example}
    Множество тензоров заданного порядка $d$ с набором мод $n_1, \ldots, n_d$ будем обозначать $\RR^N$, где $N = (n_1, \ldots, n_d) \in \NN^d$.
    К элементу тензора, стоящему на позиции $j_1, \ldots, j_d$, будем обращаться следующим образом: $\mbf{T}_{j_1, \ldots, j_d} \in \RR$.
    \begin{comment}
        Ясно, что хранение тензора в таком виде требует $\O{\max_{i = 1,\ldots, d} n_i^d}$ памяти. Число элементов тензора растёт экспоненциально
        с ростом порядка, что может быть большой проблемой, т.к. как мы увидим дельше, различные операции над тензорами могут приводить к изменению порядка.
    \end{comment}

    Пусть есть два индексных множества $N = (n_1, \ldots, n_d) \in \NN^d, M = (m_1, \ldots, m_d) \in \NN^d$. Определим множество $\RR^{M \times N}$:
    \[
        \RR^{M \times N} = \RR^{m_1 \times n_1 \times m_2 \times n_2 \times \ldots \times m_d \times n_d}
    \]
    Далее нам понадобится следующий факт: каждый тензор $\mbf{T} \in \RR^N, N = (n_1, \ldots, n_d) \in \NN^d$ можно ассоциировать с тензором
    $\overline{\mbf{T}} \in \RR^{N \times \mathtt{1}}$, $\mathtt{1} = (1, \ldots, 1) \in \NN^d$.
    \begin{example}
        Пусть $N = (3, 2, 2)$, и соответственно
        \[
            \mbf{T} = \left[
            \begin{pmatrix}
                1 & 0\\
                0 & 0\\
                1 & 1\\
            \end{pmatrix},
            \begin{pmatrix}
                2 & 2\\
                1 & 2\\
                2 & 3\\
            \end{pmatrix}\right]
            \in \RR^N
        \]
        Построим $\overline{\mbf{T}}\in \RR^{N \times \mathtt{1}} = \RR^{3 \times 1 \times 2 \times 1 \times 2 \times 1}$:
%        \[
%            \overline{\mbf{T}} =
%            \left[
%            \left[
%            \left[
%            \begin{pmatrix}
%                1\\0\\1
%            \end{pmatrix},
%            \begin{pmatrix}
%                0\\0\\1
%            \end{pmatrix}
%            \right],
%            \left[
%            \begin{pmatrix}
%                2\\1\\2
%            \end{pmatrix},
%            \begin{pmatrix}
%                2\\2\\3
%            \end{pmatrix}
%            \right]
%            \right]
%            \right]
%        \]
        \[
            \overline{\mbf{T}} =
            \left\{
                \left[
                    \left[
                        \left[
                            \begin{pmatrix}
                                1\\0\\1
                            \end{pmatrix}
                            \right],
                        \left[
                            \begin{pmatrix}
                                0\\0\\1
                            \end{pmatrix}
                            \right]
                        \right],
                    \left[
                        \left[
                            \begin{pmatrix}
                                2\\1\\2
                            \end{pmatrix}
                            \right],
                        \left[
                            \begin{pmatrix}
                                2\\2\\3
                            \end{pmatrix}
                            \right]
                        \right]
                    \right]
                \right\}
        \]
    \end{example}
    \section{Операции над тензорами}
    \subsection{Сложение и умножение на скаляр}
    Пусть есть индексное множество $N = (n_1, \ldots, n_d)$, и есть два тензора $\mbf{T}, \mbf{U} \in \RR^N$.
    \begin{definition}
        \it{Суммой тензоров $\mbf{T}, \mbf{U}$} называется тензор $\mbf{T} + \mbf{U} \in \RR^N$, задаваемый правилом
        \[
            (\mbf{T} + \mbf{U})_{j_1, \ldots, j_d} = \mbf{T}_{j_1, \ldots, j_d} + \mbf{U}_{j_1, \ldots, j_d}
        \]
    \end{definition}
    \begin{definition}
        \it{Произведением тензора $\mbf{T}$ на скаляр $\lambda$} называется тензор $\lambda \cdot \mbf{T} \in \RR^N$, задаваемый правилом
        \[
            (\lambda \cdot \mbf{T}_{j_1, \ldots, j_d}) = \lambda \cdot \mbf{T}_{j_1, \ldots, j_d}
        \]
    \end{definition}
    \begin{comment}
        Несложно показать, что множество $\RR^N$ с такими опреациями сложения и умножения является векторным пространством над полем $\RR$.
    \end{comment}
    \subsection{Тензорная свёртка}
    
\end{document}
