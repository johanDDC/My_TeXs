\section{Закон больших чисел.}
\subsection{Закон больших чисел в слабой форме.}
\begin{lemma}[Закон больших чисел в слабой форме]
    Пусть $\{X_n\}_n$ --- последовательность независимых одинаково распределённых случайных величин, и $\EE(X_n^2) < \infty$.
    Пусть $\EE(X_1) = m$, тогда $\forall\;\epsilon > 0$ верно
    \[
        \lim\limits_{n \to \infty} P\left( \left| \frac{\sum\limits_{k = 1}^n X_k}{n} - m \right| \geq \epsilon \right) = 0
    \]
\end{lemma}
\begin{proof}
    Т.к. все величины одинаково распределены и $\EE(X_1) = m$, то $\forall\; i\; \EE(X_i) = m$. Тогда
    \[
        \EE\left( \frac{\sum\limits_{k = 1}^n X_k}{n} \right) = \sum\limits_{k = 1}^n \EE\left( \frac{X_k}{n} \right) =
        n \cdot \frac{\EE(X_1)}{n} = m
    \]
    Тогда по \hyperref[ChebyshevCorollary]{следствию из неравенства Чебышева}:
    \[
        P\left( \left| \frac{\sum\limits_{k = 1}^n X_k}{n} - m \right| \geq \epsilon \right)
        \leq \frac{\DD\left(\frac{1}{n}\sum\limits_{k = 1}^n X_k\right)}{\epsilon^2} =
        \frac{\DD\left(\sum\limits_{k = 1}^n X_k\right)}{n^2\epsilon^2} =
        \frac{n\DD(X_1)}{n^2\epsilon^2} = \frac{\DD(X_1)}{n\epsilon^2}
    \]
    Устремляя $n$ к бесконечности получаем доказательство утверждения.
\end{proof}
\begin{comment}
    Пусть $X_k$ --- независимые бернулиевские случайные величины с вероятностью успеха $p$. Тогда величина
    $\frac{\sum\limits_{k = 1}^n X_k}{n}$ --- частота успешного исхода эксперимента при проведении $n$ независимых
    испытаний. Известно, что $\EE(X_k) = p$, тогда $\DD(X_k) = \EE(X_k^2) - \EE(X_k)^2 = p^2 - p = pq$, где $q = (1 - p)$.
    Тогда
    \[
        P\left( \left| \frac{\sum\limits_{k = 1}^n X_k}{n} - p \right| \geq \epsilon \right)
        \leq \frac{pq}{n\epsilon^2} \xrightarrow[n \to \infty]{} 0
    \]
    Т.е. при проведении большого числа испытаний, частота успешного результата эсперимента стремится к $p$.

    Можно разобрать смысл этого утверждения на примере: пусть эксперимент у нас состоит в подкидывании монеты, успехом
    мы считаем выпадение орла. Очевидно, что в одном независимом испытании орёл выпадает с вероятностью $\frac{1}{2}$.
    При этом в жизни подбрасывая монету $5, 10$ и даже $100$ раз мы можем ни разу не получить орла. Но данное утверждение
    говорит нам о том, что при проведении огромного числа испытаний, вероятность выпадения орла в среднем во всех испытаниях
    будет примерно $\frac{1}{2}$.
\end{comment}
\subsection{Теорема Муавра-Лапласа.}
Пусть $X_k$ --- независимые бернулиевские случайные величины с вероятностью успеха $p$. Пусть $S_n$ --- количество успешных испытаний
$\left( S_n = \sum\limits_{k = 1}^{n} X_k \right)$. $S_n$ имеет биномиальное распределение:
\[
    P(S_n = k) = \binom{n}{k}p^{k}q^{n - k}
\]
Исследуем поведение $P(S_n = k)$ при больших $n$.
\begin{theorem}[Муавра-Лапласа]
    Если $n \to \infty$, вероятность исхода $p \in (0, 1)$ фиксирована, величина $x_m = \frac{m - np}{\sqrt{npq}}$ ограничена
    равномерно по $m, n$ ($\exists\;a, b \in \RR \colon a \leq x_m \leq b$), то
    \[
        P(S_n = m) \sim \frac{1}{\sqrt{npq}}\phi(x_m)
    \]
    где $\phi(x_m) = \frac{1}{\sqrt{2\pi}}e^{-\frac{x^2}{2}}$ --- \it{плотность стандарного нормального распределения} (что
    бы это ни значило).
\end{theorem}