\section{Пример: из сходимости по вероятности не следует сходимость почти наверное.}
Построим последовательность $X_n$, такую что $X_n \toP 0$:
возьмём отрезок $[0, 1]$ и разобьём его на $2^n$ подотрезков. Пусть случайная величина
$Y_{n, k} = I_{[2^{-n}k, 2^{-n}(k+1)]}$ (индикатор попадания в $k$-ый кусочек).
Пусть последовательность $X_n$ имеет вид $Y_{1,1}, Y_{2, 1}, Y_{2, 2}, \ldots, Y_{n, 1}, \ldots, Y_{n, n}$.

Ясно, что эта последовательность сходится к 0 по вероятности. Действительно, начиная с некоторого члена,
функции становятся нулевыми на множестве, мера которого не меньше $1 - \frac{1}{n}$, что стремится к
единице. Соответственно, мера множества, на котором функция отличается от нулевой, будет стремиться к
нулю.

Вместе с тем, сходимости почти наверное (то есть на множестве меры 1) здесь нет.
Более того, ни в одной из точек отрезка последовательность функций не стремится к нулю,
так как для любой из точек, в нулевой последовательности значений функций будут время от времени
попадаться единицы (пусть всё реже и реже, но при этом бесконечное число раз).


\section{Пример: из сходимости по распределению не следует сходимость по вероятности.}
Положим $\Omega = [0, 1]$. Пусть случайные величины $X, Y$ задана следующим образом:
\begin{align*}
    &X \sim \begin{cases}
               X(t) = 1 & t \geq 0,5\\
               X(t) = 0 & t < 0,5
    \end{cases}
    &Y \sim \begin{cases}
                Y(t) = 1 & t \leq 0,5\\
                Y(t) = 0 & t > 0,5
    \end{cases}
\end{align*}
И задана последовательность $X_n = X$. Тогда ясно, что $X_n \toD X$ и $X_n \toD Y$. Но при этом
$P(X_n = Y) = 0$, откуда $X_n \toP Y$ не выполнено.

\section{Пример: Если компоненты вектора имеют нормальное распределение, то не обязательно весь
вектор нормален.}

Пусть $X \sim \mathcal{N}(0, 1), \epsilon$ --- независимая с $X$ случайная величина. Известно, что
$P(\epsilon = 1) = P(\epsilon = -1) = \frac{1}{2}$. Тогда случайная величина $Y = \epsilon X$
имеет стандартное нормальное распределение (проверяется через функцию распределения). Причём
оказывается, что $X, Y$ --- независимы.
Если рассмотреть вектор $(X, Y)^T$, то найдётся вектор $v = (1, 1)^T$, такой что
$\left( (X, Y)^T, (1, 1)^T \right)$ не имеет нормальное распределение.
Действительно:
\begin{align*}
    \left( (X, Y)^T, (1, 1)^T \right) = X + Y = X(1 + \epsilon)
\end{align*}
Эта случайная величина принимает значение $0$ с вероятностью $\frac{1}{2}$.