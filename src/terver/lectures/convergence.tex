\section{Сходимость случайных величин.}
\subsection{Закон больших чисел. (Обобщённый вариант)}
Пусть $X$ --- некоторая случайная величина и $k \in \NN$.
\begin{definition}
    \it{$k$-ым начальным моментном} случайной величины $X$ называется конечная величина
    \[
        v_k = \EE(X^k)
    \]
\end{definition}
\begin{definition}
    \it{$k$-ым центральным моментом} случайной величины $X$ называтся конечная величина
    \[
        \mu_k = \EE[(X - \EE(X))^2]
    \]
\end{definition}
\begin{proposal}(Неправенство Чебышева).\\

    Пусть у неотрицательной случайной величины $X$ определено математическое ожидание. Тогда $\forall\, t > 0$ выполнено
    \[
        P(X \geq t) \leq \frac{\EE(X)}{t}
    \]

    Пусть теперь у случайной величины $X$ определён второй начальный момент. Тогда
    \[
        P(|X - \EE(X)| \geq \epsilon) \leq \frac{\DD(X)}{\epsilon^2}
    \]
\end{proposal}