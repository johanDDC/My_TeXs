\section{Математическая статистика}
\subsection{Оценки параметров и их свойства}
\begin{example}
    Пусть есть $N$ шаров, из них $M$ шаров белого цвета. Случайным образом выбирается $n$ шаров.

    \it{Вопрос с точки зрения теории вероятности:} С какой вероятностью среди $n$ выбранных шаров окажется $m$ белых?

    \it{Вопрос с точки зрения математической статистики:} Известно, что среди $n$ выбранных шаров есть $m$ белых. Сколько всего
    есть белых шаров среди всех $N$ шаров (чему равно $M$)?
\end{example}
Основная задача математической статистики состоит в нахождении неизвестного распределения. Пусть неизвестное распределение
принадлежит какому-то фиксированному семейству распределений с функциями распределения $F_{\theta}$. Т.е. это функции распределения,
которые зависят не только от естественного аргумента, но ещё и от некоторого параметра $\theta$, который может быть произвольным,
но обладать фиксированными свойствами в рамках конкретной задачи. Обозначим множество параметром, обладающих такими свойсвами через
 $\Theta \subset \RR^n$. Тогда, если мы найдём параметр $\theta$, соответствующий нашему распределению, то это будер равносильно
нахождению самого неизвестного распределения.
\begin{example}
    Пусть известно, что искомое распределение является нормальным с дисперсией $1$, но не известно, где у него центр. Но пусть
    известно, что центр распределения неотрицателен. Тогда можно запараметризовать семейство распределений следующим образом:
    $\{\mathcal{N}(\theta, 1)\}_{\theta \geq 0}$. Параметр $\theta$ нам не известен, но в рамках задачи известны его свойства:
    $\theta$ неотрицателен. Тогда можно сказать, что $\theta$ принадлежит множеству параметров с такими свойствами $\Theta$,
    которое имеет вид луча: $\Theta = [0, \infty)$.
\end{example}
Таким образом нашей задачей является поиск параметра $\theta$.

\begin{definition}
    Случайный вектор $X = (X_1, \ldots, X_n)$ называется \it{выборкой}, если его компоненты независимы и одинаково распределены.
\end{definition}
\begin{definition}
    Все случайные величины вида $T(X)$, где $X$ --- выборка, а $T$ --- произвольная функция, называются \it{статистиками}.
\end{definition}
\begin{definition}
    Статистика, область значений которой является подмножеством в $\Theta$ называется \it{оценкой параметра} $\theta$.
\end{definition}
Пусть мы имеем выборку $X = (X_1, \ldots, X_n)$, которая является результатом проведения $n$ независимых экспериментов с
неизвестным распределением $F_{\theta}$. Имея выборку нам бы хотелось приблизить значение неизвестного параметра $\theta$
некоторым параметром $\hat{\theta}$, которое мы можем найти. Т.е. сейчас наша задача --- найти оценку $\hat{\theta}(X)$,
в некотором смысле приближающую параметр $\theta$.

Поймём теперь, в каком смысле оценка $\hat{\theta}$ будет приближать параметр $\theta$:
\begin{definition}
    Оценка $\hat{\theta}$ называется \it{несмещённой}, если $\EE_{\theta}[\hat{\theta}(X)] = \theta$.
\end{definition}
\begin{definition}
    Оценка $\hat{\theta}$ называется \it{состоятельной}, если $\hat{\theta}(X) \toP \theta$
\end{definition}
\begin{definition}
    Оценка $\hat{\theta}$ называется \it{сильно состоятельной}, если $\hat{\theta}(X) \toPN \theta$
\end{definition}
\begin{definition}
    Оценка $\hat{\theta}$ называется \it{ассимптотически нормальной}, если
    \[
        \sqrt {n}(\hat{\theta}(X) - \theta) \toD Z \sim \mathcal{N}(0, \sigma^2(\theta))
    \]
    Число $\sigma(\theta)$ называется \it{ассимптотической дисперсией}. Тако условие влечёт состоятельность и позволяет оценить
    вероятности событий вида $\alpha \leq \hat{\theta} \leq \beta$.
\end{definition}
\begin{comment}
    Обычно состоятельности следуют из законов больших чисел и теорем о непрерывности.

    Ассимптотическая нормальность следует из центральной предельной теоремы.
\end{comment}
\begin{example}
    Пусть есть множество случайных величин $\{X_j\}_{j = 1}^n$ с определённым вторым моментом, распределение которых
    параметризуется следующим образом: $X_j \sim \mathcal{N}(\theta, 1)$. Тогда
    выборочное среднее является несмещённой, сильно состоятельно оценкой матожидания:
        \[
            \overline{X_n} = \frac{X_1 + \ldots + X_n}{n} \toP \EE(X_1) = \theta
        \]
\end{example}
\begin{example}
    Пусть есть множество случайных величин $\{X_j\}_{j = 1}^n$, с определённым чеивёртым моментом, распределение которых
    параметризуется следующим образом: $X_j \sim \mathcal{N}(0, \theta), \theta > 0$. Тогда
    выборочная дисперсия является несмещённой, состоятельной оценкой дисперсии:
    \[
        s_n^2 = \frac{\sum_{j = 1}^{n} (X_j - \overline{X_n})^2}{n - 1} \toP \theta
    \]
\end{example}
\begin{definition}
    
\end{definition}